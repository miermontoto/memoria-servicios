\Chapter{Servicios de acceso remoto y transferencia}{}

\section{Acceso remoto}
\subsection{Ejercicio 1}
El ejercicio solicita que se escriba un programa \lstinline{telnet1_ejemplo.py}
que muestre por pantalla el resultado de ejecutar la orden \lstinline{ls} en el
directorio \Verb#/home#.

\begin{lstlisting}[language=Python]
    import getpass
    import telnetlib

    encoding = 'utf8'
    HOST = "localhost"
    user = input("Enter your remote account: ")
    password = getpass.getpass()

    tn = telnetlib.Telnet(HOST)

    tn.read_until(b"login: ")
    tn.write(user.encode(encoding) + b"\n")
    if password:
        tn.read_until(b"Password: ")
        tn.write(password.encode(encoding) + b"\n")

    tn.write(b"cd /home\n")
    tn.write(b"ls\n")
    tn.write(b"exit\n")

    print(tn.read_all().decode(encoding))
\end{lstlisting}

El ejercicio solicita posteriormente que se mejore el programa \\
(\lstinline{telnet1_ejemplo_mejorado.py}) para que omita el mensaje de bienvenida
del sistema.

\begin{lstlisting}[language=Python]
    import getpass
    import telnetlib

    encoding = 'utf8'
    HOST = "localhost"
    user = input("Enter your remote account: ")
    password = getpass.getpass()

    tn = telnetlib.Telnet(HOST)

    tn.read_until(b"login: ")
    tn.write(user.encode(encoding) + b"\n")
    if password:
        tn.read_until(b"Password: ")
        tn.write(password.encode(encoding) + b"\n")

    tn.read_until("$".encode(encoding))

    tn.write(b"cd /home\n")
    tn.write(b"ls\n")
    tn.write(b"exit\n")

    print(tn.read_all().decode(encoding))
\end{lstlisting}

\subsection{Ejercicio 2}

El ejercicio solicita escribir un programa \lstinline{telnet2_lanza_servidor.py}
que lance un programa de la sesión \nameref{Servicios básicos}.

\begin{lstlisting}[language=Python]
    import getpass
    import telnetlib
    import time

    encoding = 'utf8'
    HOST = "localhost"
    user = input("Enter your remote account: ")
    password = getpass.getpass()

    tn = telnetlib.Telnet(HOST)

    tn.read_until(b"login: ")
    tn.write(user.encode(encoding) + b"\n")
    if password:
        tn.read_until(b"Password: ")
        tn.write(password.encode(encoding) + b"\n")

    tn.read_until("$".encode(encoding))

    tn.write(b"ps -ef\n")

    output = tn.read_until("$".encode(encoding)).decode(encoding)

    print("udp_servidor3_con_ok.py" in output)

    if ("udp_servidor3_con_ok.py" in output):
        print("El servidor ya esta en ejecucion")
        exit(0)

    tn.write(f"nohup python3 udp_servidor3_con_ok.py &\n".encode(encoding))
    time.sleep(1)
    tn.write(b"exit\n")

    print(tn.read_all().decode(encoding))
\end{lstlisting}

\subsection{Ejercicio 3}
\subsection{Ejercicio 4}
\subsection{Ejercicio 5}
\subsection{Ejercicio 6}
\subsection{Ejercicio 7}
\subsection{Ejercicio 8}

El ejercicio solicita que se modifique el programa \lstinline{ssh_ejemplo_mal.py}
para que incluya la línea \lstinline{client.set_missing_host_key_policy(paramiko.AutoAddPolicy())}
(resultando en el \lstinline{ssh_ejemplo_inseguro.py}).

Introducimos la línea en cuestión entre la creación del cliente \Verb#paramiko#
y la conexión con el servidor ssh.

\begin{lstlisting}[language=Python]
    import paramiko
    import time
    from getpass import getpass

    user: str = input("Introduzca nombre de usuario: ")
    password: str = getpass()

    client = paramiko.SSHClient()
    client.set_missing_host_key_policy(paramiko.AutoAddPolicy())
    client.connect('localhost', username=user, password=password)
    print("Conectado!!")


    # Ejecutar comando remoto, redireccionando sus salidas
    stdin, stdout, stderr = client.exec_command('ls')

    # Mostrar resultado de la ejecución (rstrip quita los retornos de carro)
    for line in stdout:
        print(line.rstrip())
    time.sleep(1)  # Dar tiempo a que se vacie el buffer
    client.close()
\end{lstlisting}

\subsection{Ejercicio 9}
El ejercicio solicita que se modifique el programa \lstinline{ssh_ejemplo_inseguro.py}
para que el cliente utilice la \Verb#WarningPolicy#.

Modificamos la política de la línea introducida en el ejercicio anterior y
obtenemos el fichero \lstinline{ssh_ejemplo_inseguro2.py}.

\begin{lstlisting}[language=Python]
    # ... El resto del programa es igual al anterior
    client = paramiko.SSHClient()
    client.set_missing_host_key_policy(paramiko.WarningPolicy())
    client.connect('localhost', username=user, password=password)
    print("Conectado!!")
    # ... El resto del programa es igual al anterior
\end{lstlisting}

\subsection{Ejercicio 10}

El ejercicio solicita que se modifique el programa \lstinline{ssh_ejemplo_inseguro2.py},
eliminando la línea añadida previamente e insertando dos líneas nuevas para asegurar la conexión.

\begin{lstlisting}[language=Python]
    import paramiko
    import time
    import base64
    from getpass import getpass

    # ... El resto del programa es igual al anterior

    client = paramiko.SSHClient()
    key = paramiko.Ed25519Key(data=base64.decodestring(b'pubkey')) # See /etc/ssh/
    client.get_host_keys().add('localhost', 'ssh-ed25519', key)
    client.connect('localhost', username=user, password=password)
    print("Conectado!!")

    # ... El resto del programa es igual al anterior
\end{lstlisting}

\subsection{Ejercicio 11}

El ejercicio solicita que se modifique el programa \lstinline{ssh_ejemplo_seguro.py}
para que sea posible conectarse al ordenador del compañero.

\begin{lstlisting}[language=Python]
    # ... El resto del programa es igual al anterior

    host: str = input("Introduzca la dirección del host a conectarse: ")

    client = paramiko.SSHClient()
    key = paramiko.Ed25519Key(data=base64.decodestring(b'pubkey')) # See /etc/ssh/
    client.get_host_keys().add('localhost', 'ssh-ed25519', key)
    client.connect(host, username=user, password=password)
    print("Conectado!!")

    # ... El resto del programa es igual al anterior
\end{lstlisting}

\section{Transferencia de Archivos}
\subsection{Ejercicio 12}
\subsection{Ejercicio 13}
\subsection{Ejercicio 14}
\subsection{Ejercicio 15}
\subsection{Ejercicio 16}
\subsection{Ejercicio 17}
\subsection{Ejercicio 18}
\subsection{Ejercicio 19}
