\chapter{Título del capítulo 1}

\section{Nombre de la sección 1.1}

\lipsum[1-5]

\section{Nombre de la sección 1.2}

\lipsum[8]

\subsection{Nombre de la subsección 1.2.1}

\lipsum[11-12]

\subsection{Nombre de la subsección 1.2.2}

\lipsum[14-15]

\begin{itemize}
    \item elemento 1 de la lista
    \item elemento 2 de la lista
    \item elemento 3 de la lista
    \item elemento 4 de la lista
    \item elemento 5 de la lista
\end{itemize}

\lipsum[17]

\begin{enumerate}
    \item elemento 1 de la lista
    \item elemento 2 de la lista
    \item elemento 3 de la lista
    \item elemento 4 de la lista
    \item elemento 5 de la lista
\end{enumerate}

\lipsum[18]

Aquí se hace una referencia a la Figura~\ref{fig:fig1}.

\begin{figure}
    \centering
    \includegraphics[width=0.5\textwidth]{images/style/logo_epi.png}
    \caption{Este será el pie de foto de la imagen}
    \label{fig:fig1}
\end{figure}

Esta sería otra referencia, en este caso, a la Figura~\ref{fig:fig2}.

\begin{figure}
    \centering
    \includegraphics[width=\textwidth]{images/style/logo_epi.png}
    \caption{Este será el pie de foto de la imagen 2}
    \label{fig:fig2}
\end{figure}

\subsection{Nombre de la subsección 1.2.3}

\lipsum[3-6]

\subsection{Nombre de la subsección 1.2.4}

\lipsum[3-6]

\section{Nombre de la sección 1.3}

A continuación, dos ejemplos de tablas.

\begin{table}
    \centering
    \begin{tabular}{rcl}
        \hline \hline
            A & B & C \\ \hline
         1 & 2 & 3 \\
         1 & 2 & 3 \\
         1 & 2 & 3 \\
        \hline \hline
    \end{tabular}
    \caption{Titulo de la tabla 1}
    \label{tab:tab1}
\end{table}

\begin{table}
    \centering
    \begin{tabular}{lll}
        \hline \hline
            A & B & C \\ \hline
         1 & 2 & 3 \\
         1 & 2 & 3 \\
         1 & 2 & 3 \\
        \hline \hline
    \end{tabular}
    \caption{Titulo de la tabla 2}
    \label{tab:tab2}
\end{table}

Es importante ver como, al incluir tablas correctamente etiquetadas éstas son añadidas en el índice de tablas. Lo mismo ocurre con las imágenes.
De igual forma que con las imágenes, también se puede hacer una referencia a la Tabla~\ref{tab:tab1} o la Sección~\ref{sec:sec21} si se coloca la etiqueta label correspondiente.

\chapter{Titulo del capitulo 2}
\section{Nombre de la sección 2.1}
\label{sec:sec21}

\lipsum[56-58]

\section{Nombre de la sección 2.2}

Para hacer un uso adecuado de la bibliografía, solamente habrá que citar los artículos que correspondan \cite{guerreiro2017mismatch,ministerio_energia,benedetti2018anomaly,antonanzas2016review,darksky}. Pero también se podrán citar de uno en uno como este \cite{garoudja2017enhanced} y \cite{belaout2018multiclass}.

Para hacer estas citas, será necesario el fichero .bib con la información de cada cita, que se puede encontrar en casi todos los repositorios virtuales.
