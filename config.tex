% Se define un color gris desde su código RGB
\definecolor{gris}{RGB}{220,220,220}

\setcounter{secnumdepth}{3} % Para permitir numerar las sub-subsecciones

% Modifica el nombre de los índices al castellano
\addto\captionsspanish{
  \renewcommand{\contentsname}{Índice de contenido}
  \renewcommand{\listfigurename}{Índice de figuras}
  \renewcommand{\listtablename}{Índice de tablas}
}

% Formateo de los nombres de los apartados:
\titleformat{\chapter}[block]
  {\normalfont\Huge\bfseries\singlespacing}{\thechapter.}{1em}{\Huge}
\titlespacing*{\chapter}{0pt}{-62pt}{0pt}

\titleformat{\section}[block]
  {\normalfont\Large\bfseries}{\thesection.}{4pt}{\Large}
\titlespacing*{\section}{0pt}{\baselineskip}{0pt}

\titleformat{\subsection}[block]
  {\normalfont\large\bfseries}{\thesubsection.}{4pt}{\normalsize\large}
\titlespacing*{\subsection}{0pt}{0pt}{0pt}

\titleformat{\subsubsection}[block]
  {\normalfont\normalsize\bfseries}{\thesubsubsection.}{4pt}{\normalsize}
\titlespacing*{\subsubsection}{0pt}{0pt}{0pt}

\def\tablename{Tabla}

%% Variables para portada y cabeceras
%% Cambiar los valores para cada documento!!!
\def\title{Memoria de Prácticas}
\def\subject{Ingeniería de Servicios}
\def\authorOne{Juan Francisco Mier Montoto}
\def\authorOneId{UO283319}
\def\authorTwo{Alejandro Rodríguez López}
\def\authorTwoId{UO281827}
\def\group{PL 3,~grupo 1}
\def\date{enero 2024}
\def\org{Escuela Politécnica de Ingeniería de Gijón}
\def\area{Grado en Ingeniería Informática en Tecnologías de la Información}

\def\ORG{\expandafter\MakeUppercase\expandafter{\org}}
\def\AREA{\expandafter\MakeUppercase\expandafter{\area}}
\def\SUBJECT{\expandafter\MakeUppercase\expandafter{\subject}}
\def\authors{\authorOne{,}~\authorTwo}

\captionsetup{justification=centering}
\setlength{\headheight}{65pt}

\fancyhf{}
\fancyhead[L]{\includegraphics[height=16mm]{style/square.png}
  \hspace{1em} \Longstack[l] {
    \textbf{\SUBJECT} \newline
    \textbf{\title}}
  \newline \leftmark{}
}
\fancyhead[R]{\bfseries{Hoja \hyperlink{toc}{\thepage}~de~\pageref{LastPage}}}
\fancyfoot[C]{\authors}
\renewcommand{\headrulewidth}{0pt} % default is 0pt
\renewcommand{\footrulewidth}{0.4pt} % default is 0

\fancypagestyle{plain}{%
  \fancyhf{}
  \fancyhead[L]{\includegraphics[height=16mm]{style/square.png}
    \hspace{1em} \Longstack[l]{
      \textbf{\SUBJECT} \newline
      \textbf{\title}}}
  \fancyhead[R]{\bfseries{Hoja \hyperlink{toc}{\thepage}~de~\pageref{LastPage}}}
  \fancyfoot[C]{\authors}
  \renewcommand{\headrulewidth}{0pt} % default is 0pt
  \renewcommand{\footrulewidth}{0.4pt} % default is 0pt
}

\pagestyle{fancy}

\restylefloat{table}

\definecolor{background}{HTML}{EEEEEE}
\definecolor{delim}{RGB}{20,105,176}
\definecolor{lightgray}{rgb}{.9,.9,.9}
\definecolor{darkgray}{rgb}{.4,.4,.4}
\definecolor{purple}{rgb}{0.65, 0.12, 0.82}
\definecolor{green}{rgb}{0.15, 0.62, 0.22}

\lstset{ % General lstlisting parameters
   backgroundcolor=\color{lightgray},
   extendedchars=true,
   basicstyle=\footnotesize\ttfamily,
   showstringspaces=false,
   showspaces=false,
   numbers=left,
   numberstyle=\footnotesize,
   numbersep=9pt,
   tabsize=2,
   breaklines=true,
   showtabs=false,
   captionpos=b
}

\lstdefinelanguage{Python}{ % Python specific lstlisting parameters
  keywords={def, typeof, new, True, False, try, except, return, null, if, elif, in, for, while, do, else, break},
  morekeywords=[1]{,as,assert,nonlocal,with,yield,self,True,False,None,}, % Python builtin
  morekeywords=[2]{,__init__,__add__,__mul__,__div__,__sub__,__call__,__getitem__,__setitem__,__eq__,__ne__,__nonzero__,__rmul__,__radd__,__repr__,__str__,__get__,__truediv__,__pow__,__name__,__future__,__all__,}, % magic methods
  morekeywords=[3]{,print,fit,iloc,loc,forecast,predict,from,dropna,astype,transform,describe,display,summary,show,plot,int,float,bool,Dict,Callable,any,object,type,isinstance,copy,deepcopy,zip,enumerate,reversed,list,set,len,dict,tuple,range,xrange,append,execfile,real,imag,reduce,str,repr,}, % common functions
  morekeywords=[4]{,Exception,NameError,IndexError,SyntaxError,TypeError,ValueError,OverflowError,ZeroDivisionError,}, % errors
  morekeywords=[5]{,ode,fsolve,sqrt,exp,sin,cos,arctan,arctan2,arccos,pi, array,norm,solve,dot,arange,isscalar,max,sum,flatten,shape,reshape,find,any,all,abs,plot,linspace,legend,quad,polyval,polyfit,hstack,concatenate,vstack,column_stack,empty,zeros,ones,rand,vander,grid,pcolor,eig,eigs,eigvals,svd,qr,tan,det,logspace,roll,min,mean,cumsum,cumprod,diff,vectorize,lstsq,cla,eye,xlabel,ylabel,squeeze,}, % numpy / math
  keywordstyle=\color{blue}\bfseries,
  ndkeywords={class, export, boolean, throw, implements, import, this},
  ndkeywordstyle=\color{green}\bfseries,
  identifierstyle=\color{black},
  sensitive=false,
  comment=[l]{\#},
  morecomment=[s]{/*}{*/},
  commentstyle=\color{purple}\ttfamily,
  stringstyle=\color{red}\ttfamily,
  morestring=[b]',
  morestring=[b]"
}

% I got these from: https://raw.githubusercontent.com/marekaf/docker-lstlisting/master/latex.tex

\lstdefinelanguage{docker}{
  keywords={FROM, RUN, COPY, ADD, ENTRYPOINT, CMD,  ENV, ARG, WORKDIR, EXPOSE, LABEL, USER, VOLUME, STOPSIGNAL, ONBUILD, MAINTAINER, HEALTHCHECK},
  keywordstyle=\color{blue}\bfseries,
  identifierstyle=\color{black},
  sensitive=false,
  comment=[l]{\#},
  commentstyle=\color{purple}\ttfamily,
  stringstyle=\color{red}\ttfamily,
  morestring=[b]',
  morestring=[b]"
}

\lstdefinelanguage{docker-compose}{
  keywords={image, environment, ports, container_name, ports, volumes, links},
  keywordstyle=\color{blue}\bfseries,
  identifierstyle=\color{black},
  sensitive=false,
  comment=[l]{\#},
  commentstyle=\color{purple}\ttfamily,
  stringstyle=\color{red}\ttfamily,
  morestring=[b]',
  morestring=[b]"
}
\lstdefinelanguage{docker-compose-2}{
  keywords={version, volumes, services},
  keywordstyle=\color{blue}\bfseries,
  keywords=[2]{image, environment, ports, container_name, ports, links, build},
  keywordstyle=[2]\color{olive}\bfseries,
  identifierstyle=\color{black},
  sensitive=false,
  comment=[l]{\#},
  commentstyle=\color{purple}\ttfamily,
  stringstyle=\color{red}\ttfamily,
  morestring=[b]',
  morestring=[b]"
}

\newcommand{\bold}[1]{\textbf{#1}\ }
\newcommand{\italic}[1]{\textit{#1}\ }

\newcommand\Chapter[2]{
  \chapter[#1: {\itshape#2}]{#1\\\strut\hfill\Large\itshape#2}
  \label{#1}
}

\usepackage{tcolorbox}
\newtcolorbox{notebox}{
  colback=white,
  colframe=black,
  boxrule=1pt,
  arc=4pt,
  fonttitle=\bfseries,
  title=NOTA
}
